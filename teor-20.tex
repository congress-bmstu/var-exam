\que{Задача со свободными концами, условия трансверсальности (с доказательством).}

Среди всех функций $y(x)$ на отрезке $[a, b]$ найти ту, которая даёт экстремум функционалу
\[
  J(y) = \int\limits_a^b F(x, y, y') \, dx
\]

Для того, чтобы некоторая кривая обеспечивала экстремум, она должна быть экстремалью, то есть
удовлетворять уравнению Эйлера. Другим необходимым условием является выполнение
равенств
\[
  F_{y'} |_{x=a} = 0, \quad
  F_{y'} |_{x=b} = 0,
\]
называемых \emph{условиями трансверсальности}.

\begin{proof}
  (взято из Гельфанд-Фомин, стр. 31.)
  Вычислим вариацию этого функционала:
  \[
    J(y+h) - J(y) = \int\limits_a^b [ F(x, y+h, y' + h') - F(x, y, y') ] \, dx
    = \int\limits_a^b [ F_y h + F_{y'} h' ] \, dx + o,
  \]
  таким образом:
  \[
    \delta J = \int\limits_a^b [F_y h + F_{y'} h'] \, dx
  \]
  Так как здесь, в отличие от задачи с закрепленными концами, $h(x)$ уже не обязательно
  обращается в нуль в точках $a$ и $b$, то, интегрируя по частям, получаем:
  \[
    \delta J =
    \int\limits_a^b [F_y - \dfrac{d}{dx} F_{y'}] h(x) \, dx + [F_{y'} \delta y] |_{x=a}^{x=b} = 
    \int\limits_a^b [F_y-\dfrac{d}{dx} F_{y'}] h(x) \, dx + F_{y'} |_{x=b} h(b) - F_{y'} |_{x=a} h(a)
  \]

  Рассмотрим сначала такие функции $h(x)$, для которых $h(a) = h(b) = 0$, тогда, как и в
  простейшей задаче, из условия $\delta J = 0$ получаем, что
  \[
    F_y - \dfrac{d}{dx} F_{y'} = 0.
  \]
  Пусть теперь $y = y(x)$ -- экстремаль, тогда в выражении для $\delta J$ интегральный член
  исчезает и условие $\delta J = 0$ принимает вид
  \[
    F_{y'} |_{x=b} \, h(b) - F_{y'} |_{x=a} \, h(a) = 0,
  \]
  т.е., в силу произвольности $h(x)$,
  \[
    F_{y'} |_{x=b} = 0, \quad F_{y'} |_{x=a} = 0.
  \]

\end{proof}
