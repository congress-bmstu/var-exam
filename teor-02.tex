\que{Золотое сечение, формулы золотого сечения (с доказательством). Одномерная оптимизация: метод золотого сечения.}

\paragraph{Метод золотого сечения. } 
\begin{definition}
	\textit{Золотым сечением} отрезка $[x, y]$ называется деление отрезка на две неравные части так, чтобы отношение длины всего отрезка к длине большей части равнялось отношению длины большей части к длине меньшей части. 
\end{definition}

\begin{utv}
	Золотое сечение отрезка производится двумя точками $u_1$ и $u_2$, которые расположены симметрично относительно отрезка и $u_1 = x + \frac{3 - \sqrt{5}}{2} (y - x) \approx x + 0.382 (y - x), u_2 = x + \frac{\sqrt{5} - 1}{2} (y - x) \approx x + 0.618 (y - x)$.
\end{utv}
\begin{proof}
	\begin{align*}
		&\frac{y - x}{y - u_1} = \frac{y - u_1}{u_1 - x}, \quad (y - x) (u_1 - x) = (y - u_1)^2; \\ &y u_1 - y x - x u_1 + x^2 = y^2 - 2 y u_1 + u_1^2,
		u_1^2 + (-3 y + x) u_1 + y^2 + x y - x^2 = 0; \\ &D = (-3 y + x)^2 - 4 y^2 - 4 xy + 4 x^2 = 5 y^2 - 10 xy + 5 x^2 = 5 (y - x)^2.
	\end{align*}
	Имеем: $u_{1, 2} = \frac{3 y - x \pm (y - x) \sqrt{5}}{2}$. Тогда:
	$\frac{1}{2} (3 y - x + \sqrt{5} (y - x)) = y + \frac{1}{2} (\underbrace{(y - x)}_{>0} \underbrace{(1 + \sqrt{5})}_{>0}) \Rightarrow$ эта точка не подходит.
	
	Подходит другой корень, который и будет: $u_1 = \frac{1}{2} (3 y - x - \sqrt{5} (y - x)) = \frac{1}{2} (3 y - x - \sqrt{5} (y - x)) = \\ = x + \frac{1}{2} (3 y - 3 x - \sqrt{5}(y - x)) = x + \frac{1}{2} (y - x) (3 + \sqrt{5})$.
\end{proof}

\begin{utv}
	Пусть точки $u_1$ и $u_2$ производят золотое сечение отрезка $[x, y]$. Тогда точка $u_1$ производит золотое сечение отрещка $[x, u_2]$, а $u_2$ --- отрезка $[u_1, y]$.
\end{utv}

\textit{Теперь же изложим алгоритм.} Пусть функция $f(x)$ унимодальная на $[a, b]$ положим $a_1 = a, \, b_1 = b$. Найдем $u_1$ и $u_2$, образующие золотое сечение $[a, b]$.

Вычислим $f(u_1)$ и $f(u_2)$.
\begin{enumerate}[label={\arabic*)}]
	\item Если $f(u_1) \leqslant f(u_2) \Rightarrow a_2 = a_1, \, b_2 = u_2, \bar{u}_2 = u_1$.
	
	\item Если $f(u_1) > f(u_2) \Rightarrow a_2 = u_1, \, b_2 = b_1, \bar{u}_2 = u_2$.
\end{enumerate}

Получаем $f(x)$ унимодальна на $[a_2, b_2]$. $\Omega_{\ast} \cap [a_2, b_2] \not = \emptyset; l([a_2, b_2]) = \frac{\sqrt{5} - 1}{2} (b - a) < l([a_1, b_1])$.
\newline

\textit{Опишем $n$-й шаг}. 

$[a_{n - 1}, b_{n - 1}], \, \Omega_{\ast} \cap [a_{n - 1}, b_{n - 1}] \not = \emptyset, \, l([a_{n - 1}, b_{n - 1}]) = \left(\frac{\sqrt{5} - 1}{2}\right)^n (b - a)$.

Определены точки $u_1, u_2, \dotsc, u_{n - 1}$. Вычислены $f(u_1), f(u_2), \dotsc, f(u_{n - 1})$ и имеется $\bar{u}_{n - 1}$.

Возьмем $u_n: u_n \not = \bar{u}_{n - 1}$ и производит золотое сечение отрезка $[a_{n - 1}, b_{n - 1}]$. Вычислим $f(u_n)$:
\begin{enumerate}[label={\arabic*)}]
	\item $f(u_2) \leqslant f(\bar{u}_{n - 1}) \Rightarrow a_n = a_{n - 1}, b_n = \bar{u}_{n - 1}, \bar{u}_n = u_n$;
	
	\item $f(u_n) > f(\bar{u}_{n - 1}) \Rightarrow \bar{u}_n = u_n, a_n = u_n, b_n = b_{n - 1}, u_n = \bar{u}_{n - 1}$. 
\end{enumerate}

Остальные случаи аналогичны. 
\newline

\textit{Условия остановки алгоритма:}
\begin{enumerate}
	\item $l([a_n, b_n]) < \varepsilon$, где $\varepsilon$ --- заранее выбранное число, $\varepsilon > 0$;
	
	\item если проведенно заданное число итераций. 
\end{enumerate}

\textit{Точка минимума} $\bar{u}_n, m_{\ast} \approx f(\bar{u}_n)$.