\que{Афинное пространство. Отрезок в аффинном пространстве. Выпуклое множество. Отрезок и аффинное многообразие, как выпуклые множества (с доказательствами). Вершина выпуклого множества.}

\begin{definition}
	Пусть $V$ --- векторное пространство над полем $\mathbb{R}$. \textit{Аффинным пространством}, ассоциированным с векторным пространством $V$, называется множество $S$ с операцией сложения и удовлетворяющее следующим аксимомам:
	\begin{enumerate}[label={(\arabic*)}]
		\item $(\forall p \in S) (\forall x, y \in V): \, p + (x + y) = (p + x) + y$;
		
		\item $(\forall p \in S): p + \bar{0} = p$;
		
		\item $(\forall p, q \in S) (\exists! x \in V): p + x = q.$
	\end{enumerate}
\end{definition}


\begin{definition}
	Пусть $S$ --- афинное пространство, ассоциированное с веторным пространством $V$, $p, q \in S$.
	
	\textit{Отрезком}, соединяющим точки $p$ и $q$ называется множество $[p, q] = \{x \in S \mid x = \lambda p + (1 - \lambda) q, \lambda \in [0, 1] \subset \mathbb{R}\}$.
\end{definition}

\begin{definition}\label{выпуклое множество}
	Подмножество $T$ аффинного пространства $S$ называется \textit{выпуклым множеством}, если оно вместе с любыми двумя своими точками $p, q$ содержит и отрезок $[p, q]$:
	\begin{equation*}
		(\forall p, q \in T) \quad [p, q] \subset T.
	\end{equation*}
	
	\begin{figure}[H]
		\centering 
		\includesvg[scale=0.8]{conv}
	\end{figure}
	
	$\emptyset$ --- выпуклое множество.
	$p, q \in V [p, q]$ --- выпуклое множество, состоящее из одного отрезка.
\end{definition}

\begin{utv}
	Отрезок в аффинном пространстве $S$ является выпуклым множеством.
\end{utv}
\begin{proof}
	$p, q \in S, [p, q]; \quad y, z \in [p, q] \Rightarrow \begin{matrix}
		y = \lambda_1 p + (1 - \lambda_1) q \\
		z = \lambda_2 p + (1 - \lambda_2) q
	\end{matrix}, \quad \lambda_1, \lambda_2 \in \mathbb{R}; \lambda_1, \lambda_2 \in [0, 1]$. 
	
	Рассмотрим $x = \lambda y + (1 - \lambda) z \in [y, z]$ и произвольную точку $o \in S: \overline{ox} = \lambda \overline{oy} + (1 - \lambda) \overline{oz}$.
	
	Тогда:
	\begin{align*}
		\overline{ox} &= \lambda (\lambda_1 \overline{op} + (1 - \lambda_1) \overline{oq}) + (1 - \lambda) (\lambda_2 \overline{op} + (1 - \lambda_2) \overline{oq}) = \\ &= \underbrace{(\lambda \lambda_1 + (1 - \lambda) \lambda_2)}_{\mu_1} \overline{op} + \underbrace{(\lambda (1 - \lambda_1) + (1 - \lambda) (1 - \lambda_2))}_{\mu_2} \overline{oq}; \\
	\end{align*}
	
	$\lambda, \lambda_1, \lambda_2 \in \mathbb{R} \Rightarrow \mu_1, \mu_2 \in \mathbb{R};$. Нужно доказать, что коэффициенты $\mu_1, \mu_2 \in [0, 1]$. 
	
	\begin{align*}
		\mu_1 + \mu_2 &= \lambda \lambda_1 + (1 - \lambda) \lambda_2 + \lambda (1 - \lambda_1) + (1 - \lambda) (1 - \lambda_2) = \\ &= \underbrace{(\lambda_1 + 1 - \lambda_1)}_{1} \lambda + \underbrace{(\lambda_2 + 1 - \lambda_2)}_{1} (1 - \lambda) = \lambda + 1 - \lambda = 1; \\
	\end{align*}
	
	$\Rightarrow \mu_1 = 1 - \mu_2$, и, т.к. $\mu_1 \in [0,1]$, то и $\mu_2 \in [0, 1]$. 
	
	Таким образом, $x = \mu_1 p + (1 - \mu_1) q \in [p, q]$. 
\end{proof}

\begin{definition}
	$S$ --- аффинное пространство, ассоциированное с $V$ $p \in S$, $U \leqslant V: $
	\begin{equation*}
		A = p + U = \{p + u \mid u \in U\} 
	\end{equation*}
	назыввается \textit{аффинным многообразием или ($m$-мерной) плоскостью в пространстве $S$}.
\end{definition}

\begin{utv}
	Аффинное многообразие является выпуклым множеством. 
\end{utv}
\begin{proof}
	$A = p + U; \, q, r \in A, \, x \in [q, r] \Rightarrow u = \lambda q + (1 - \lambda) r, \, \lambda \in \mathbb{R}, \, \lambda \in [0, 1], o \in S$.
	\begin{align*}
		&\overline{ox} = \lambda \overline{oq} + (1 - \lambda) \overline{or}; \\
		&q = p + u_1, \, r = p + u_2, \, u_1, u_2 \in U,\text{т.е. } \overline{oq} = \overline{op} + u_1; \, \overline{or} = \overline{op} + u_2; \\
		\Rightarrow &\overline{ox} = \lambda (\overline{op} + u_1) + (1 - \lambda) (\overline{op} + u_2) = (\lambda + (1 - \lambda)) \overline{op} + (\lambda u_1 + (1 - \lambda) u_2) = \\ = &\overline{op} + \underbrace{(\lambda u_1 + (1 - \lambda) u_2)}_{= u, u \in U} = \overline{op} + u \Rightarrow x = p + u \in A. 
	\end{align*}
\end{proof}

\begin{definition}
	$S$ --- аффинное пространство, $M$ --- выпуклое множество. Элемент $x \in M$ \textit{вершиной} множества $M$, если его нельзя представить в виде $x = \lambda p + (1 - \lambda) q$, где $p, q (p \not = q) \in M, \lambda \in [0, 1]$. Т.е., $x = q = p$. 
\end{definition}