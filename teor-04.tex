\que{Условие Липшица, свойства постоянной Липшица (с доказательством). Одномерная оптимизация: метод ломанных.}

\begin{definition}
	Говорят, что функция $f(x): [a, b] \to \mathbb{R}$ удоввлетворяет условию Липшица на $[a, b]$, если $\exists L > 0: \abs{f(u) - f(v)} \leqslant L \cdot \abs{u - v}$ для всех $u, v \in [a, b]$. 
	
	Константу $L$ называют \textit{постоянной Липшица} для функции $f(x)$ на $[a, b]$.
\end{definition}

\begin{utv}
	Указанное условие имеет простой геометрический смысл: угловой коэффициент (тангенс угла наклона) прямой, соединяющей точки $(u, f(u)), (v, f(v))$ не превышает $L$ для всех точек $u, v \in [a, b]$. 
	Если функция $f(x)$ удовлетворяет условию Липшица на $[a, b]$, то она непрерывна на этом отрезке. Докажем последнее. 
\end{utv}
\begin{proof}
	$u \in [a, b]$.
	
	Функция непрерывна, если $(\forall \varepsilon > 0) (\exists \delta > 0) (\forall v \in U_{\delta}(u)): f(v) \in U_{\delta}(f(u))$. Возьмем $\delta = \frac{\varepsilon}{L}$.
	
	$v \in U_{\delta}(u) \cap [a, b]$. 
	
	$\abs{f(u) - f(v)} \leqslant L \cdot \abs{u - v} < L \cdot \delta = L \cdot \frac{\varepsilon}{L} = \varepsilon$. 
\end{proof}

\begin{theorem} \label{LipshForAll}
	Пусть функция $f(x)$ непрерывна на отрезке $[a, b]$ и на каждом из отрезков $[x_i, x_{i + 1}]$, где $i = 1, \dotsc, n$, $a = x_1 < x_2 \dotsc < x_n = b$, удовлетворяет условию Липшица с постоянной $L_i$. Тогда функция $f(x)$ удовлетворяет условию Липшица на отрезке $[a, b]$ с постоянной $L = \max(L_i)_{i = 1, \dotsc, n}$. 
\end{theorem}
\begin{proof}
	Возьмём произвольные точки $u, v \in [a, b]$. Тогда $x_{p - 1} \leqslant u \leqslant x_p, x_q \leqslant v < x_{q + 1}$ для некоторых $p, q$. Можно без ограничения общности считать, что $p \leqslant q$. Получаем:
	\begin{align*}
		&\abs{f(u) - f(v)} = \abs{(f(u) - f(x_p)) + \sum_{i = p}^{q - 1}(f(x_i) - f(x_{i + 1})) + (f(x_q) - f(v))} \leqslant \\
		&\leqslant \abs{f(u) - f(x_p)} + \sum_{i = p}^{q - 1}(\abs{f(x_i) - f(x_{i + 1})}) + \abs{f(x_q) - f(v)} \leqslant \\
		&\leqslant L_{p - 1} \abs{u - x_p} + \sum_{i = p}^{q - 1}(L_i \abs{x_i - x_{i + 1}}) + L_q \abs{x_q - v} \leqslant \\
		&\leqslant L\left(u - x_p + \sum_{i = p}^{q - 1}(x_i - x_{i + 1}) + x_q - v\right) = L\abs{u - v}. 
	\end{align*}
\end{proof}

\begin{theorem}
	Пусть функция $f(x)$ дифференцируема на отрезке $[a, b]$ и её производная $f'(x)$ ограничена на этом отрезке. Тогда функция $f(x)$ удовлетворяет условию Липшица с постоянной $L = \sup{\abs{f'(x)}}_{x \in [a, b]}$.
\end{theorem}
\begin{proof}
	По теореме Лагранжа для любых $u, v \in [a, b]$ имеем 
	\begin{equation*}
		\abs{f(u) - f(v)} = \abs{f'(\xi)} \cdot \abs{u -v}
	\end{equation*}
	для некоторой точки $\xi$, лежащей между $u$ и $v$. Отсюда $\abs{f(u) - f(v)} \leqslant L\abs{u - v}$. 
\end{proof}

\begin{utv}
	Пусть функция $f(x) : [a, b] \to \mathbb{R}$ удовлетворяет условию Липшица с постоянной $L$ на отрезке $[a, b]$. Зафиксируем точку $v \in [a, b]$. Рассмотрим функцию $g_{v}(x) = f(v) - L\abs{x - v}$. То есть
	\begin{equation*}
		g_v(x) = \begin{cases}
			f(v) - L(v - x) = L x + f(v) - L v, \quad \text{ если } x \leqslant v, \\
			f(v) - L(x - v) = - L x + f(v) + L v, \quad \text{ если } x > v.
		\end{cases}
	\end{equation*}
	Таким образом, график функции $g_v(x)$ на отрезке $[a, b]$ представляет собой ломаную линию, состоящую из отрезков двух прямых, имеющих угловые коэффициенты $L$ и $-L$ (то есть первая прямая возрастает, а вторая убывает), и пересекающихся в точке $(v, f(v))$. 
	
	Кроме того, $g_v(v) = f(v)$ и для $x \not = v$
	\begin{equation*}
		f(x) - g_v(x) = f(x) - (f(v) - L\abs{x - v}) = f(x) - f(v) + L\abs{x - v} \geqslant L \abs{x - v} - \abs{f(x) - f(v)} \geqslant 0. 
	\end{equation*}
	
	Получаем, что график функции $f(x)$ расположен не ниже графика функции $g_v(x)$ во всех точках $x \in [a, b], x \not = v$ и имеет с ним общую точку $(v, f(v))$.
\end{utv}

\paragraph{Метод ломанных. } Данный метод работает для так называемых многомодальных функций, то есть теперь мы не будем требовать, чтобы исследуемая функция была унимодальной. 

Рассмотрим функцию $f(x) : [a, b] \to \mathbb{R}$, удовлетворяющую условию Липшица с постоянной $L$. Выберем произвольную точку $x_0 \in [a, b]$ и составим функцию $g_{x_0}(x) = p_0(x)$. Найдём точку $x_1$ из условия
\begin{equation*}
	x_1 \in [a, b], p_0(x_1) = \min{p_0(x)}_{x \in [a, b]}.
\end{equation*} 

Ясно, что $x_1 = a$ или $x_1 = b$.

Возьмем теперь функцию 
\begin{equation*}
	p_1(x) = \max(g_{x_1}(x), p_0(x)).
\end{equation*}

Точку $x_2$ найдём из условий 
\begin{equation*}
	x_2 \in [a, b], p_1(x_2) = \min{p_1(x)}_{x \in [a, b]}. 
\end{equation*}

Пусть точки $x_0, x_1, \dotsc, x_n, n \geqslant 1$ уже известны. Тогда составим функцию
\begin{equation*}
	p_n(x) = \max(g_{x_n}(x), p_{n - 1}(x)) = \max{g_{x_1}(x)}_{o \leqslant i \leqslant n}.
\end{equation*}

Следующая точка $x_{n + 1}$ определяется из условий 
\begin{equation*}
	x_{n + 1} \in [a, b], p_n(x_{n + 1}) = \min{p_n(x)}_{x \in [a, b]}.
\end{equation*}

Если минимум достигается в нескольких точках, то можно взять любую из них. 

Из теоремы \ref{LipshForAll} следует, что функция $p_n(x)$ удовлетворяет условию Липшица с той же постоянной $L$, что и $f(x)$. Также для всех $x \in [a, b]$
\begin{equation*}
	p_{n - 1}(x) = \max{g_{x_i}(x)}_{0 \leqslant i \leqslant n - 1} \leqslant \max{g_{x_i}(x)}_{0 \leqslant i \leqslant n} = p_n(x).
\end{equation*}

Кроме того, $g_i(x) \leqslant f(x)$ для всех $x \in [a, b], \, i = 0, 1, \dotsc, n$, отсюда $p_n(x) \leqslant f(x)$ для всех $n$ и $x \in [a, b]$.

Таким образом, на каждом шаге метода ломанных задача минимизации функции $f(x)$ заменяется более простой задачей минимизации кусочно-линейной функции $p_n(x)$, которая приближает $f(x)$ снизу, причём значения $\{p_n(x)\}$ в каждой точке $x$ с ростом $n$ монотонно возрастают. Можно доказать, что данный метод сходится.
