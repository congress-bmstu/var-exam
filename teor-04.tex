\que{Условие Липшица, свойства постоянной Липшица (с доказательством). Одномерная оптимизация: метод ломанных.}

\paragraph{Метод ломанных. } Данный метод работает для так называемых многомодальных функций, то есть теперь мы не будем требовать, чтобы исследуемая функция была унимодальной. 

\begin{definition}
	Говорят, что функция $f(x): [a, b] \to \mathbb{R}$ удоввлетворяет условию Липшица на $[a, b]$, если $\exists L > 0: \abs{f(u) - f(v)} \leqslant L \cdot \abs{u - v}$ для всех $u, v \in [a, b]$. 
	
	Константу $L$ называют \textit{постоянной Липшица} для функции $f(x)$ на $[a, b]$.
\end{definition}

\begin{utv}
	Указанное условие имеет простой геометрический смысл: угловой коэффициент (тангенс угла наклона) прямой, соединяющей точки $(u, f(u)), (v, f(v))$ не превышает $L$ для всех точек $u, v \in [a, b]$. 
	Если функция $f(x)$ удовлетворяет условию Липшица на $[a, b]$, то она дифференцируема на этом отрезке. 
\end{utv}
