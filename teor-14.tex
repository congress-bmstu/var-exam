\que{Нормированные и метрические пространства. Связь нормированных и метрических пространств. Пространства $\mathbb{R}^n, C[a, b], D_1[a, b]$. Норма и метрика в пространствах $\mathbb{R}^n, C[a, b], D_1[a, b]$ (с доказательством).}

\begin{definition}
	\textit{Нормой} называется неотрицательная вещественнозначная функция, определённая на векторном пространстве $V$, и обладающая свойствами
	\begin{enumerate}
		\item $\norm{v} = 0 \Rightarrow v = 0$,
		
		\item $\norm{\lambda v} = \abs{\lambda} \cdot \abs{v}$ (\textsc{однородность}),
		
		\item $\norm{u + v} \leqslant \norm{u} + \norm{v}$ (\textsc{неравенство треугольника})
	\end{enumerate}
	для любых $u, v \in V$.
	
	Векторное пространство с определённой нормой называют \textit{нормированным}.
\end{definition}

\begin{definition}
	\textit{Метрикой} называется неотрицательная вещественнозначная фнукция двух аргументов $\rho(x, y): A \to \mathbb{R}$, определённая на произвольном множестве $A$, со свойствами 
	\begin{enumerate}
		\item $\rho(x, y) = 0 \Leftrightarrow x = y$,
		
		\item $\rho(x, y) = \rho(y, x)$ (\textsc{симметричность}),
		
		\item $\rho(x, z) \leqslant \rho(x, y) + \rho(y, z)$ (\textsc{неравенство треугольника})
	\end{enumerate}
	для любых $x, y, z \in A$.
	
	Множество с введённой метрикой называют \textit{метрическим пространством}.
\end{definition}

\begin{utv}
	Любое нормированное пространство является метрическим, еслли определить метрику $\rho(u, v) = \norm{u - v}$.
\end{utv}
\begin{proof}
	\begin{enumerate}
		\item $\rho(u, v) = \norm{u - v} \geqslant 0$,
		
		$\rho(u, v) = 0 \Leftrightarrow \norm{u - v} = 0 \Leftrightarrow u - v = 0 \Leftrightarrow u = v$;
		
		\item $\rho(u, v) = \norm{u - v} = \norm{u - w + w - v} \leqslant \norm{u - w} + \norm{w - v} = \rho(u. w) + \rho(w, v)$;
		
		\item $\rho(u, v) = \norm{u, v} = \norm{(- 1) (v - u)} = \abs{- 1} \cdot \norm{v - u} = \norm{v - u} = \rho(v, u)$. 
	\end{enumerate}
\end{proof}

\begin{definition}
	$C[a, b]$ --- пространство непрерывных на отрезке $[a, b]$ функций, имеет норму:
	\begin{equation*}
		\norm{y}_0 = \max\limits_{a \leqslant x \leqslant b}{\abs{y(x)}}.
	\end{equation*}
\end{definition}
\begin{utv}
	Легко убедиться, что $\norm{y}_0$ действительно норма в $C[a, b]$.
\end{utv}
\begin{proof}
	\begin{enumerate}
		\item $\norm{y}_0 = \max\limits_{a \leqslant x \leqslant b}{\abs{y(x)}} \geqslant 0 \text{, поскольку } \abs{y(x)} \geqslant 0 \text{ для всех } x$, \\ 
		$\norm{y}_0 = \max\limits_{a \leqslant x \leqslant b}{\abs{y(x)}} = 0 \text{ тогда и только тогда, когда } 0 \leqslant \abs{y(x)} \leqslant 0 \text{ для любого } x \in [a, b], \text{ то есть }$ \\  $y(x) \equiv 0;$ 
		
		\item $\norm{\alpha y}_0 = \max\limits_{a \leqslant x \leqslant b}{\abs{(\alpha y)(x)}} = \max\limits_{a \leqslant x \leqslant b}{\abs{\alpha \cdot y(x)}} = \max\limits_{a \leqslant x \leqslant b}{\abs{\alpha}\cdot\abs{y(x)}} = \abs{\alpha} \cdot \max\limits_{a \leqslant x \leqslant b}{\abs{y(x)}} = \abs{\alpha} \cdot \norm{y}_0$;
			
		\item $\norm{y + z}_0 = \max\limits_{a \leqslant x \leqslant b}{\abs{(y + z)(x)}} = \max\limits_{a \leqslant x \leqslant b}{\abs{y(x) + z(x)}} \leqslant \max\limits_{a \leqslant x \leqslant b}{(\abs{y(x)} + \abs{z(x)})} \leqslant \max\limits_{a \leqslant x \leqslant b}{\abs{y(x)}} + + \max\limits_{a \leqslant x \leqslant b}{\abs{z(x)}} = \norm{y}_0 + \norm{z}_0$.
	\end{enumerate}
\end{proof}

\begin{corollary}
	В $C[a, b]$ определена метрика $\rho(y_1, y_2) = \norm{y_1 - y_2}_{0} = \max\limits_{a \leqslant x \leqslant b}{\abs{y_1(x) - y_2(x)}}$.
\end{corollary}

Тогда мы можем определить $\varepsilon$-окрестность произвольной точки $y_0$ относительно метрики $\rho_0$. А именно, 
\begin{equation*}
	U_{0, \varepsilon} = \{y \in C[a, b] \mid \rho_0(y_0, y) < \varepsilon\} = \{y \in C[a, b] \mid \abs{y_0(x) - y(x)} < \varepsilon \text{ для всех } x \in [a, b]\}.
\end{equation*}

Геометрически это означает, что в $\varepsilon$-окрестности функции $y_0$ лежат все непрерывные на отрезке $[a, b]$ функции, графики которых на всём отрезке $[a, b]$ лежат внутри полосы шириной $2 \varepsilon$  по вертикали вокруг графика функции $y_0$, то есть в каждой точке на $\varepsilon$ вверх и на $\varepsilon$ вниз от значения функции $y_0$ в данной точке. 

Про такие функции будем говорить, что они близки в смысле близости 0-го порядка. 

\begin{definition}
	$D_1[a, b]$ --- пространство всех непрерывно дифференцируемых функций на отрезке $[a, b]$, имеет норму:
	\begin{equation*}
		\norm{y}_1 = \max\limits_{a \leqslant x \leqslant b}{\abs{y(x)}} + \max\limits_{a \leqslant x \leqslant b}{\abs{y'(x)}}.
	\end{equation*}
\end{definition}

\begin{utv}
	Аналогично предыдущему утверждению, можем проверить, что это действительно норма в $D_1[a, b]$.
\end{utv}
\begin{proof}
	\begin{enumerate}
		\item Очевидно;
		
		\item $\norm{\alpha y}_1 = \max\limits_{a \leqslant x \leqslant b}{\abs{(\alpha y)(x)}} + \max\limits_{a \leqslant x \leqslant b}{\abs{(\alpha y')(x)}} = \max\limits_{a \leqslant x \leqslant b}{\abs{\alpha \cdot y(x)}} + \max\limits_{a \leqslant x \leqslant b}{\abs{\alpha \cdot y'(x)}} = \abs{\alpha} \cdot \max\limits_{a \leqslant x \leqslant b}{\abs{y(x)}} + + \abs{\alpha} \cdot \max\limits_{a \leqslant x \leqslant b}{\abs{y'(x)}} = \abs{\alpha} \cdot \left(\max\limits_{a \leqslant x \leqslant b}{\abs{y(x)}} + \max\limits_{a \leqslant x \leqslant b}{\abs{y'(x)}}\right) = \abs{\alpha} \cdot \norm{y}_1$;
		
		\item $\norm{y + z}_0 = \max\limits_{a \leqslant x \leqslant b}{\abs{(y + z)(x)}} + \max\limits_{a \leqslant x \leqslant b}{\abs{(y' + z')(x)}} \leqslant \max\limits_{a \leqslant x \leqslant b}{\abs{y(x) + y'(x) + z(x) + z'(x)}} \leqslant \\ \leqslant \max\limits_{a \leqslant x \leqslant b}{(\abs{y(x)} + \abs{y'(x)} + \abs{z(x)} + \abs{z'(x)})} \leqslant \left(\max\limits_{a \leqslant x \leqslant b}{\abs{y(x)}} + \max\limits_{a \leqslant x \leqslant b}{\abs{y'(x)}}\right) + \\ + \left(\max\limits_{a \leqslant x \leqslant b}{\abs{z(x)}} + \max\limits_{a \leqslant x \leqslant b}{\abs{z'(x)}}\right) = \norm{y}_1 + \norm{z}_1$.
	\end{enumerate}
\end{proof}

\begin{corollary}
	В пространстве $D_1[a, b]$ также определена метрика $\rho_1(y_1, y_2) = \norm{y_1 - y_2}_1 = = \max\limits_{a \leqslant x \leqslant b}{\abs{y_1(x) - y_2(x)}} + \max\limits_{a \leqslant x \leqslant b}{\abs{y_1'(x) - y_2'(x)}}$.
\end{corollary}

Тогда $\varepsilon$-окрестность произвольной точки $y_0$ относительно метрики $\rho_1$
\begin{equation*}
	U_{1, \varepsilon} = \{y \in D_1[a, b] \mid \abs{y_0(x) - y(x)} < \varepsilon \text{ и } \abs{y'_0(x) - y'(x)} < \varepsilon \text{ для всех } x \in [a, b]\}.
\end{equation*}

Геометрически это означает, что в $\varepsilon$-окрестности функции $y_0$ лежат все непрерывно дифференцируемые на отрезке $[a, b]$ функции, графики которых на всём отрезке $[a, b]$ лежат внутри полосы шириной $2 \varepsilon$ по вертикали вокруг графика функции $y_0$, и производные данных функций, то есть тангенсы угла наклона касательных, отличаются от производной функции $y_0$ в каждой точке не более, чем на $\varepsilon$.

Про такие функции будем говорить, что они близки в смысле близости 1-го порядка.

\begin{utv}
	Если две функции $y_1, y_2 \in D_1[a, b]$ и близки в смысле близости 1-го порядка, то они также близки и в смысле близости 0-го порядка, то есть если $\rho_1(y_1, y_2) < \varepsilon$ для некоторого $\epsilon > 0$, то также $\rho_0(y_1, y_2 < \varepsilon)$.
\end{utv}
\begin{utv}
	Согласно предыдущему утверждению, можно сформулировать еще одно определение окрестности точки, для $y_0 \in D_1[a, b]$. 
	
	Назовём $\varepsilon$-окрестностью элемента $y_0$ пространства $D_1[a, b]$ в смысле метрики $\rho_0$ множество 
	\begin{align*}
		&U_{0, \varepsilon}(y_0) = \{y \in D_1[a, b] \mid \rho_0(y_0, y) < \varepsilon\} = \{y \in D_1[a, b] \mid \abs{y_0(x) - y(x)} < \varepsilon \text{ для всех } x \in [a, b]\}.
	\end{align*} 
	
	Ясно, что тогда
	\begin{equation*}
		U_{1, \varepsilon}(y_0) \subset U_{0, \varepsilon}(y_0).
	\end{equation*}
\end{utv}

