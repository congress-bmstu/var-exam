\que{Выпуклые тройки, свойства выпуклых троек. Интерполяция (с доказательством). Одномерная оптимизация: метод парабол.}

\begin{definition}
	Пусть задана $f(x): [a, b] \to \mathbb{R}$. Тройку $x_1, x_2, x_3 \in [a, b]$ назовем выпуклой тройкой для функции $f(x)$, если $a \leqslant x_1 < x_2 < x_3 \leqslant b$; $f(x_1) \geqslant f(x_2), f(x_2) \leqslant f(x_3), f(x_1) + f(x_3) > 2 f(x_2)$. 
\end{definition}

\begin{utv}
	Пусть точки $x_1, x_2, x_3$ образуют выпуклую тройку для $f(x)$.
  Тогда точки плоскости $(x_1, f(x_1)), (x_2, f(x_2))$ и $(x_3, f(x_3))$ не лежат на одной прямой и
  через них можно провести параболу ветвями вверх. 
\end{utv}
\begin{proof}
  \begin{enumerate}
    \item Не лежание на одной прямой очевидно. 
    \item Через любые три точки (попарно различные), можно провести параболу. 
      Чтобы ветви были вверх, достаточно условия $ f(x_1) + f(x_3) > 2 f(x_2)$.
  \end{enumerate}
\end{proof}

\paragraph{Метод парабол.} Пусть функция $f(x)$ --- унимодальная на $[a, b]$ Вберем шаг $h > 0$, где $2 h \leqslant b - a$ и начальную точку $u_0$.

Рассмотрим точку $u_1 = u_0 + h$. Если $u_1 \in [a, b]$, то вычислим $f(u_1)$. Если $f(u_1) \leqslant f(u_0)$, то вычисляем $u_i = u_0 + 2^{i - 1} \cdot h, i \geqslant 2$. 

Проверим, что $u_i \in [a, b]$. Если да, вычислим $f(u_i)$. Теперь, проверяем образуют ли $u_{i - 2}, u_{i - 1}, u_i$ выпуклую тройку для $f(x)$.

Если не образуют, то возьмем следующую точку. Если найдена выпуклая тройка, то проведем через нее парабол ветвями вверх. 

Найдем точку $w$ --- точку минимума параболы (вершина параболы). 

Если $u_i \not \in [a, b]$, а выпуклая тройка так и не найдена, тогда положим $w = b$. 

Если $f(u_1) > f(u_0)$ или $u_1 \in [a, b]$, тогда изменим направление потока. Переобозначим $u_1$ на $u_0$: $u_i = u_0 - 2^{i - 1} \cdot h, i \geqslant 1$.

Если тройка найдена, $w$ --- вершина параболы, если нет, то $w = a$. 

Далее, $f(\bar{u}) = \min(f(w), f(u_0), \dotsc, f(u_n))$.
