\que{Линейное программирование, основная и каноническая задачи. Геометрическая интерпретация (с обоснованием).}

% Источник: ВИКИПЕДИЯ https://ru.wikipedia.org/wiki/%D0%9B%D0%B8%D0%BD%D0%B5%D0%B9%D0%BD%D0%BE%D0%B5_%D0%BF%D1%80%D0%BE%D0%B3%D1%80%D0%B0%D0%BC%D0%BC%D0%B8%D1%80%D0%BE%D0%B2%D0%B0%D0%BD%D0%B8%D0%B5

\paragraph{Основная задача линейного программирования.}
Ищется экстремум линейной функции $f(\vec{x}) = \sum_{i=0}^N c_i x_i \to \text{extr}$.
С ограничениями типа неравенств:
\[
  \begin{cases}
    \sum_{i=0}^N a^{(1)}_i x_i \geqslant b^{(1)}, \\
    \sum_{i=0}^N a^{(2)}_i x_i \geqslant b^{(2)}, \\
    \dots \\
    \sum_{i=0}^N a^{(m)}_i x_i \geqslant b^{(m)}, \\
    x_i \geqslant 0.
  \end{cases}
\]

\paragraph{Каноническая задача линейного программирования.}
Ищется экстремум линейной функции $f(\vec{x}) = \sum_{i=0}^N c_i x_i \to \text{extr}$.
С ограничениями типа равенств:
\[
  \begin{cases}
    \sum_{i=0}^N a^{(1)}_i x_i = b^{(1)}, \\
    \sum_{i=0}^N a^{(2)}_i x_i = b^{(2)}, \\
    \dots \\
    \sum_{i=0}^N a^{(m)}_i x_i = b^{(m)}, \\
    x_i \geqslant 0.
  \end{cases}
\]

Основную задачу всегда можно привести к канонической путём введения новых переменных:
\[
  \begin{cases}
      \sum_{i=1}^N a_i x_i \geqslant b, \\
      x_i \geqslant 0.
  \end{cases}
  \Leftrightarrow
  \begin{cases}
    \sum_{i=1}^N a_i x_i + x_{N+1} = b, \\
    x_i \geqslant 0, \\
    x_{N+1} \geqslant 0.
  \end{cases}
\]

\paragraph{Геометрическая интерпретация.}
Нарисуем область пространства, задаваемую ограничениями основной задачи. Эта область всегда
получится выпуклым множеством (возможно, неограниченным) -- внутренностью некоторого
многогранника.

Так как оптимизируемая функция является линейной, в каждой точке градиент (и, соотвественно,
антиградиент) одинаковый. Градиент показывает направление наибольшего увеличения функции, поэтому
в направлении, ортогональном к градиенту, функция не будет меняться, тогда можно построить
линии уровня, т.е. прямые (плоскости) $f(\vec{x}) = \operatorname{const}$.
Причём если рассмотреть конкретную линию уровня, то относительно неё в одну сторону функция будет
увеличиваться, в другую -- уменьшаться. Несложно сделать вывод, что экстремумы находятся тогда 
на границе рассматриваемой области, причём есть несколько вырожденных случаев, например, когда
одна из границ параллельна линиям уровня -- тогда решениями будут все точки этой границы.
Также возможен случай, когда область неограничена -- тогда один из типов экстремума будет
отсутствовать.

