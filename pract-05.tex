\subsection{Линейное программирование --- симплекс-метод.}
\begin{example}
	Рассмотрим задачу из семинара:
	
%	\paragraph{Дано.}
	\begin{gather*}
		F = - 10 x_1 - x_2 \to \min, \quad F + 10 x_1 + x_2 = 0, \\
		\begin{cases}
			2 x_1 + 11 x_2 \leqslant 7, \\
			4 x_1 - 5 x_2 \geqslant 5, \\
			x_1, x_2 \geqslant 0,
		\end{cases} \quad 
		\begin{cases}
			2 x_1 + 11 x_2 + x_3 = 7, \\
			-4 x_1 + 5 x_2 + x_4 = -5, \\
			x_1, x_2, x_3, x_4 \geqslant 0.
		\end{cases}
	\end{gather*}
	
	Симплекс-таблица:
	\begin{table}[H]
		\centering
		\begin{tabular}{|c|c|c|c|c|c|c|}
			\hline
			Базисные переменные & {$x_1$} & {$x_2$} & {$x_3$} & {$x_4$} & Свободные переменные & {Для действий} \\ \hline
			{$x_3$} & \boxed{2} & 11 & 1 & 0 & 7 & \boxed{7 : 2 = 3.5} \\
			{$x_4$} & -4 & 5 & 0 & 1 & -5 & -5 : (-4) = 1.25 \\ \hline
			{$F$} & \boxed{10} & 1 & 0 & 0 & 0 &  \\ \hline 
			{$x_1$} & 1 & $\frac{11}{2}$ & $\frac{1}{2}$ & 0 & $\frac{7}{2}$ &  \\
			{$x_4$} & 0 & 27 & 2 & 1 & 9 & \\ \hline
			{$F$} & 0 & -54 & -5 & 0 & -35 &  \\ \hline 
		\end{tabular}
	\end{table}
	
	Опишем таблицу поэтапно:
	\begin{enumerate}
		\item Выбрали первый положительный элемент строки $F$ и соответствующий столбец $x_1$ (хотим, чтобы все элементы строки $F$ были отрицательными). Первый элемент --- 10;
		
		\item Теперь необходимо поделить элементы столбца свободных переменных на соответствующие элементы из выбранного столбца. Выбираем ту строку, где результат этого вычисления максимальный --- в этой задаче это строка $x_3$;
		
		\item Переименовываем строку $x_3$ на называние выбранного столбца $x_1$. А так же как в методе Гаусса приводим строки к такому виду, чтобы первый элемент строки был равен единицы, а все под ним нулю;
		
		\item Тогда получаем, что $F_{\min} = -35$ (столбец свободных переменных). При этом $x_1 = \frac{7}{2}, x_2 = 0$.
	\end{enumerate}
	
\end{example}

\begin{example}
	Рассмотрим еще одну задачу из семинара для примера:
	
%	\paragraph{Дано.}
	\begin{gather*}
		F = -3 x_1 - 2 x_2 \to \min, \quad F + 3 x_1 + 2 x_2 = 0, \\
		\begin{cases}
			4 x_1 + 3 x_2 \leqslant 12, \\
			4 x_1 + x_2 \leqslant 8, 
			4 x_1 - x_2 \leqslant 9,
			x_1, x_2 \geqslant 0,
		\end{cases}
		\quad 
		\begin{cases}
			4 x_1 + 3 x_2 + x_3 = 12, \\
			4 x_1 + x_2 + x_4 = 8, \\
			4 x_1 - x_2 + x_5 = 9, \\
			x_1, \dotsc, x_5 \geqslant 0.
		\end{cases}
	\end{gather*}
	
	Симплекс-таблица:
	\begin{table}[H]
		\centering
		\begin{tabular}{|c|c|c|c|c|c|c|c|}
			\hline
			Базисные переменные & $x_1$ & $x_2$ & $x_3$ & $x_4$ & $x_5$ & Свободные переменные & Для действий \\ \hline
			$x_3$ & 4 & 3 & 1 & 0 & 0 & 12 & 12 : 4 = 3 \\
			$x_4$ & \boxed{4} & 1  & 0 & 1 & 0 & 8  & 8 : 4 = 2 - $\min$ \\
			$x_5$ & 4 & -1 & 0 & 0 & 1 & 9  & 9 : 4 = 2.25 \\ \hline
			$F$   & \boxed{3} & 2  & 0 & 0 & 0 & 0 & \\ \hline
			$x_3$ & 0 & \boxed{2} & 1 & -1 & 0 & 4 & $4 : 2 = 2 - \min$ \\
			$x_1$ & 1 & $\frac{1}{4}$ & 0 & $\frac{1}{4}$ & 0 & 2 & $2 : \frac{1}{4} = 8$ \\
			$x_5$ & 0 & -2 & 0 & -1 & 1 & 1 & $1 : (-2) = 0.5$ \\\hline
			$F$   & 0 & \boxed{\frac{5}{4}} & 0 & $-\frac{3}{4}$ & 0 & -6 &  \\\hline
			$x_2$ & 0 & 1 & $\frac{1}{2}$ & $-\frac{1}{2}$ & 0 & 2 &  \\
			$x_1$ & 1 & 0 & $-\frac{1}{8}$ & $\frac{3}{8}$ & 0 & $\frac{3}{2}$ &  \\
			$x_5$ & 0 & 0 & 1 & -2 & 1 & 5 &  \\\hline
			$F$   & 0 & 0 & $-\frac{5}{8}$ & $-\frac{1}{8}$ & 0 & $-\frac{17}{2}$ & \\ \hline
		\end{tabular}
	\end{table}

	Рассуждения аналогичные предыдущему примеру. Получаем, что $F_{\min} = -\frac{17}{2}, x_1 = \frac{3}{2}, x_2 = 2$.
\end{example}
