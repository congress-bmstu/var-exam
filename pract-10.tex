\subsection{Необходимые и достаточные условия слабого экстремума простейшей задачи вариационного исчисления (уравнение Эйлера, усиленное условие Лежандра, уравнение Якоби).}
\begin{example}
  Исследовать задачу на наличие слабого минимума или слабого максимума. 
  \[
    \int\limits_{0}^{1} yy' - (y')^2\,dx \to \mathrm{extr}, \quad y(0) = 0,
    \quad y(1) = 1.
  \]
\begin{solution}
Необходимым условием слабого минимума (максимума) является удовлетворение
функцией
уравнению Эйлера.


  Запишем уравнение Эйлера для подынтегральной функции $ F $.  
  \[
    2y'' = 0, \quad \text{откуда } y = C_1 x + C_2.
  \]
 Учитывая граничные условия, $ C_2 = 0 $, $ C_1 = 1 $. Значит, $ y = x $. 
 % (Ниже
 % выяснится, что решение уравнения Эйлера можно было и не искать.)

Достаточным условием слабого минимума (максимума) для допустимой экстремали является 
\begin{enumerate}
  \item выполнение усиленного условия Лежандра $ F_{y'y'} > 0 $ ($ F_{y'y'} < 0
    $),
  \item выполнение усиленного условия Якоби, то есть
    \begin{enumerate}
      \item существование решения уравнения Якоби,
      \item необратимость в нуль решения уравнения Якоби на полуинтервале $ (a,
        b]$.
    \end{enumerate}
\end{enumerate}


 Проверим усиленное условие Лежандра. Рассмотрим функцию $ F_{y'y'} = -2 < 0 $. Поскольку она всегда строго
 меньше нуля, то возможен слабый максимум. Запишем уравнения Якоби 
 \[
   \begin{cases}
     \left( F_{yy} - \frac{d}{dx} \left( F_{yy'} \right)  \right) u - \frac{d}{dx}
     \left( F_{y'y'} u' \right) = 0,\\
     u(a) = 0,\\
     u'(a) = 1,
   \end{cases}
 \]
 где $ a = 0 $ в нашем случае. В нашем случае также $ F_{yy} = 0 $, $
 \frac{d}{dx} F_{yy'} = 0 $ и $ \frac{d}{dx}F_{y'y'} = -2u'' $. Тогда  
 \[
     \begin{cases}
       2u'' =0, \\
       u(0) = 0,\\
       u'(0) = 1.
     \end{cases}
 \]
Отсюда снова $ u = x $. Для того чтобы функция $ y(x) $ была слабым максимумом,
теперь достаточно, чтобы $ u(x) $ не обращалась в нуль на полуинтервале $ (a, b] =
(0, 1] $. Наша функция удовлетворяет этому условию, поэтому \fbox{$ y(x) = x $
является слабым максимумом.} 

% Если бы $ u(x) $ обращалась в нуль на указанном полуинтервале в точке $ x_0 $,
% то эту точку мы бы назвали \emph{сопряжённой к $ a $ точкой}.
% Поскольку решение уравнения Якоби существует, то
% функция $ y $ действительно может являеться слабым максимумом. 
\end{solution}
\end{example}

