\que{Антиградиент функции нескольких переменных. Основное свойство антиградиента (с доказательством). Градиентные методы многомерной оптимизации.}

Пусть функция $f(u), u \in U \subseteq \mathbb{R}^n$ непрерывно дифференцируема в области $U$, то есть функция $f(u)$ дифференцируема, а все её частные производные непрерывны в каждой точке $U$. Хорошо известно, что необходимым условием экстремума функции в точке является равенство нулю градиента функции. Точки, в которых градиент функции равен нулю, называется стационарными. Неравный нулю градиент указывает направление наискорейшего возрастания функции. Известно также, что если $f(u)$ выпуклая функция, то в стационарной точке всегда достигается минимум. Рассмотрим методы оптимизации функции нескольких переменных, связанные с этими свойствами градиента. Пусть извесно, что в некоторой области функция имеет единственную стационарную точку и эта точка является точкой минимума. 

\begin{definition}
	Вектор $-\mathrm{grad}{(f(u))} = - f'(u)$, противоположный градиенту функции в данной точке, называется \textit{антиградиентом}. 
\end{definition}

\begin{utv}
	\begin{equation*}
		-f'(u) = \left(-\frac{\partial f}{\partial x_1}, -\frac{\partial f}{\partial x_2}, \dotsc, -\frac{\partial f}{\partial x_n}\right).
	\end{equation*}
\end{utv}

\begin{utv}
	Ясно, что ненулевой антиградиент указывает направление наибыстрейшего убывания функции в точке. 
\end{utv}

\begin{proof}
	? %TODO: Опять доказательство
\end{proof}

\textit{Идея метода:}

$u_0$ --- начальная точка.

$u_{k + 1} = u_k - \alpha_k \cdot f'(u_k)$, где $k = 0, 1, 2, \dotsc$

Число $\alpha_k$ называется длиной шага, или шагом градиентного метода.

Если $f'(u_k) = 0$, то это стационарная точка, если $f'(u_k) \not = 0$, то можно выбрать $u_{k + 1}, f(u_{k + 1}) < f(u_k)$.

Процесс продолжается, пока не:
\begin{enumerate}
	\item выполнено заданное количество итерация;
	
	\item $\abs{f'(u_{k + 1})} < \varepsilon$, где $\varepsilon > 0$ наперед заданное число.
\end{enumerate}

\paragraph{Градиентный метод дробления шага. } Выберем $\alpha_0 > 0$. На каждом шаге проверяем условие монотонности $f(u_{k + 1}) < f(u_k)$. Если оно выполняется, продолжаем $\alpha_1 = \alpha_0, \alpha_2 = \alpha_0, \dotsc$

Если условие не выполняется $f(u_{k + 1}) \leqslant f(u_k) \Rightarrow$ новая $\alpha_k = \frac{\alpha_k}{m}, \, m \in \mathbb{N}$ и продолжить опять.

\begin{example}
	Рассмотрим функцию $f(x, y) = x^2 + 2 y^2$. Положим $u_0 = (1, 2), \alpha_0 = 0.1$. Тогда
	\begin{equation*}
		f(u_0) = 9, f'(u) = (2 x, 4 y), f'(u_0) = (2, 8) \not = (0, 0), u_1 = (1, 2) - 0.1 \cdot (2, 8) = (0.8, 1.2).
	\end{equation*}
	
	Получаем
	\begin{equation*}
		f(u_1) = 3.52 < 9 = f(u_0).
	\end{equation*}
	
	Далее 
	\begin{equation*}
		f'(u_1) = (1.6, 4.8) \not = (0, 0), u_2 = (0.64, 0.72), f(u_2) = 1.45 < 3.52 = f(u_1).
	\end{equation*}
	
	Приблизительное значение точки минимума возьмём $(0.64, 0.72)$ минимум функции $m_{\ast} \approx 1.45$.
\end{example}

\paragraph{Градиентный метод наискорейшего спуска. } Опишем $(k + 1)$-й шаг. Луч $V = \{u \in \mathbb{R}^n \mid u = u_{k} - \alpha \cdot f'(u_k), \alpha \geqslant 0\}$. 

Рассмотрим $\varphi_k(\alpha) = f(u_k - \alpha \cdot f'(u_k)), \alpha \geqslant 0$. $\alpha_k: \varphi(\alpha_k) = \inf\limits_{\alpha \geqslant 0}{\varphi_k(\alpha)}, \alpha_k > 0$.

Найдем $\alpha_k$ точно или приближенно методом одномерной оптимизации. 

\begin{example}
	Рассмотрим функцию $f(x, y) = x^2 + 2 y^2$ и пусть $u_0 = (1, 2)$. Получаем $\varphi_0(\alpha) = f(u_0 - \alpha \cdot f'(u_0)) = f((1, 2) - \alpha \cdot (2, 8)) = (1 - 2 \alpha)^2 + 2(2 - 8 \alpha)^2 = 132 \alpha^2 - 68 \alpha + 9$.
	
	Минимум функции $\varphi_0(\alpha)$ достигается при $\alpha = \frac{68}{2 \cdot 132} = 0.26$. Тогда 
	\begin{equation*}
		u_1 = (1, 2) - 0.26 \cdot (2, 8) = (0.48, - 0.08), \, f(u_1) \approx 0.24, \, f'(u_1) = (0.96, -0.32).
	\end{equation*}
	
	Далее
	\begin{equation*}
		\varphi_1(\alpha) = f(u_1 - \alpha \cdot f'(u_1)) = f((0.48, -0.08) - \alpha \cdot (0.96m -0.32)) \approx 1.13 \alpha^2 - 1.02 \alpha + 0.24.
	\end{equation*} 
	
	Минимум функции $\varphi_1(\alpha)$ достигается при $\alpha = \frac{1.02}{2 \cdot 1.13} \approx 0.45$. Отсюда 
	\begin{equation*}
		u_2 = (0.48, -0.08) - 0.45 \cdot (0.96, -0.32) \approx (0.05, 0.06), \, f(u_2) \approx 0.01, \, f'(u_2) \approx (0.1, 0.24).
	\end{equation*}
	
	Приблизительное значение точки минимума возьмём $(0.05, 0.06)$, минимум функции $m_{\ast} \approx 0.01$.
\end{example}