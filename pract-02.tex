\subsection{Выпуклые функции, критерий выпуклости.}
\begin{definition}
  \emph{Главным минором} порядка $ k $ матрицы $ H = (h_{ij}) $ называется
  определитель вида 
  \[
    \Delta_{i_1\ldots i_k} = \det \begin{pmatrix}
      a_{i_1i_1} & \cdots & a_{i_1i_k} \\
      \vdots & \ddots & \vdots \\
      a_{i_ki_1} & \cdots & a_{i_ki_k}
    \end{pmatrix},
  \]
 Иными словами, при подсчёте минора порядка $ k $ мы просто оставляем в матрице
 $ k $ строк и $ k $ столбцов с одними и теми же индексами $ i_1, $ $ i_2 $, $ \ldots $, $ i_k $.
\end{definition}

\begin{example}
  Найти значения параметра $ a $, при которых функция 
  \[
      f(x, y) = ax^2 + (2a-2)xy + (a-3)y^2
  \]
  является выпуклой.
  \begin{solution}
    \textbf{1. Найти матрицу Гессе.} Для составление матрицы Гессе нужно найти
    производные второго порядка: 
    \begin{gather*}
    \begin{aligned}
      \frac{\partial f}{\partial x} &= 2ax + (2a-2)y, & \frac{\partial^2
      f}{\partial x^2} &=2a, \\
      \frac{\partial f}{\partial y} &= (2a-2)x + 2(a-3) y, & \frac{\partial^2
      f}{\partial y^2} &= 2(a-3) = 2a - 6,
    \end{aligned} \\ 
    \frac{\partial^2 f}{\partial y \partial x} = 2a -2.
  \end{gather*}
  Тогда матрица Гессе будет иметь вид 
  \[
      H = \begin{pmatrix}
        2a & 2(a-1) \\
        2(a-1) & 2(a-3)
      \end{pmatrix}.
  \]

  \textbf{2. Вычислить главные миноры (всех порядков) матрицы Гессе.} Имеем $
  \Delta_1 = 2a $, $ \Delta_2 = 2a -6 $; $ \Delta_{12} = \det H = -4(a+1) $.

  \textbf{3. Из неотрицательности миноров находим условие на $ a $.} Получаем
  систему  
  \[
      \begin{cases}
        2a \geqslant 0, \\
        2a - 6 \geqslant 0 \Rightarrow a \geqslant 3, \\
        a + 1 \leqslant 0.
      \end{cases}
  \]
  Таким образом \fbox{решений у задачи нет.} 
    
  \end{solution}
\end{example}
