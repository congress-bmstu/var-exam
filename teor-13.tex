\que{Сильно выпуклые функции. Минимум сильно выпуклой непрерывной функции (с
доказательством).} 
\subsubsection{Необходимые сведения}
\begin{definition}
  Функция $ f(x) $ называется \emph{полунепрерывной снизу} в точке $ x $, если для любой
  последовательности $ x_n \to x $ выполняется  
  \[
    \varliminf f(x_k) \geqslant f(x).  
  \]
  Функция называется полунепрерывной снизу, если полунепрерывна снизу в каждой
  точке области определения.
\end{definition}
\begin{definition}
  \emph{Множеством Лебега} называется множество 
  \[
    M(c) = \{x\colon f(x) \leqslant c\}.
  \]
\end{definition}
\begin{lemma}
  Пусть область определения $ U $ функции $ f(x) $ замкнуто. Функция $ f(x) $ полунепрерывна снизу тогда и только тогда, когда множество Лебега $
  M(c) $ замкнуто для любой точки $ c $.
\end{lemma}
\begin{proof}
  \textsl{Необходимость.} Без ограничения общности будем считать, что $ c $
  принадлежит области значений. Пусть $ x_0 $ --- предельная для $ M(c) $ точка и $
  x_k \to x_0 $, $ x_k \in M(c) $. Тогда $ f(x_0) \leqslant \varliminf f(x_k)
  \leqslant c $, откуда $ x_0 \in M(c) $.

  \textsl{Достаточность.} Пусть $ x_k \to x_0 $, а $ \varepsilon > 0 $ ---
  произвольное число. Обозначим также $ \tilde y_0 := \varliminf f(x_k) = \lim
  f(x_k) $, где без ограничения
  общности\footnote{Ведь $ \varliminf\limits_{k\to\infty} f(x_k) =
  \lim\limits_{m\to\infty} f(x_{k_m}) $.} считаем последовательность $ f(x_k) $ неубывающей. Тогда $ x_k
  \in M(\tilde y_0 + \varepsilon)$ для всех достаточно больших $ k $. Значит, и
  $ f(x) \leqslant \tilde y_0 + \varepsilon $. Ввиду произвольности $
  \varepsilon $ имеем $ f(x) \leqslant \tilde y_0 $.
\end{proof}

\begin{theorem}
  Пусть область определения $ U $ полунепрерывной снизу функции $ f(x) $ есть
  непустое замкнутое
  множество. Пусть, кроме того, для некоторой точки $ v $ множество $ M(v) $
  ограничено.

  Тогда $ \inf f > -\infty $, множество $ \{x\colon f(x) = \inf f\} $
  непусто и компактно.
\end{theorem}
\begin{proof}
  Множество $ M(v) $ замкнуто, значит, компактно. Значит\footnote{Здесь
    используется тривиальное обобщение теоремы Вейерштрассе о непрерывной на
  компакте функции.}, функция $ f(x)|_{M(v)}
  $ в некоторой точке $ x_m \in M(v) $ принимает своё минимальное значение $ y_m
  $. То же, очевидно, касается и всей функции $ f(x) $.
\end{proof}

\subsubsection{Сильно выпуклые функции}
\begin{definition}
  Функция $ f(x) $, определённая на выпуклом множестве $ U $, называется
  \emph{сильно выпуклой} функцией на этом множестве, если существует такая
  постоянная $ \varkappa > 0$, что 
  \[
      f(\alpha u + (1-\alpha)v) \leqslant \alpha J(u) + (1-\alpha)J(v) -
      \alpha(1-\alpha)\varkappa|u-v|^2
  \]
 при всех $ u, v \in U $ и $ 0\leqslant \alpha \leqslant 1 $. Постоянную $
 \varkappa $ называют \emph{постоянной сильной выпуклости}. 
\end{definition}
Понятно, что из сильной выпуклости следует (строгая) выпуклость. Смысл этого
дополнительного ограничения в том, что с ним функции обязательно достигают своей
нижней грани.

\textsc{Пример.} Функция $ f(x) = x^2 = |x|^2 $ с $ \varkappa = 1 $.
Действительно, заменив для краткости $ \beta := 1- \alpha $, получим
\[
  (\alpha u + \beta v)^2 + \alpha \beta (u -v)^2 = \alpha^2 u^2 + \beta^2 v^2 +
  \alpha\beta u^2 + \alpha \beta v^2= \alpha u^2 + \beta v^2.
\]
В данном частном случае получили знак равенства.

\begin{theorem}
  Пусть область определения $ U $ сильно выпуклой и полунепрерывной функции $
   f(x) $ выпукла и замкнута. Тогда $ \inf f > -\infty $, а множество $
   \{x\colon f(x) = \inf f\} $ состоит ровно из одной точки.
\end{theorem}
\begin{proof}
 Интерес предоставляет случай, когда множество $ U $ неограничено. Возьмём
 произвольную точку $ v $ и единичный замкнутый шар $ B(v; 1) $. На нём функция
 $ f(x) $ принимает минимальное значение $ y'_{\min} $. Введём обозначение $
 \iota := f(v) - y'_{\min} \geqslant 0 $. Зафиксируем также некоторую точку $ u $ не
 из шара и обозначим $  0 < \alpha_0 := 1/|u-v| < 1 $. Полагая $ \alpha =
 \alpha_0 $, получаем  
 \[
     \alpha_0 f(u) \geqslant f(v + \alpha_0(u-v)) - (1-\alpha_0)f(v) +
     \alpha_0(1-\alpha_0)\varkappa|u-v|^2.
 \]
Но $ \alpha_0 |u-v| = 1 $, откуда $ v + \alpha_0(u-v) \in B $ и $
f(v+\alpha_0(u-v)) \geqslant f(v) - \iota $. Тогда  
\[
    f(u) \geqslant f(v) + (1-\alpha_0)\varkappa|u-v|^2 - \frac{\iota}{\alpha_0} = f(v) +
    \varkappa|u-v|^2-(|u-v|\sqrt\varkappa)(\varkappa + \iota/\sqrt\varkappa).
\]
Применим неравенство $ ab \leqslant (a^2+ b^2)/2 $ к последнему слагаемому: 
\begin{equation}\label{eq:7}
  f(u)\geqslant f(v) + \frac{\varkappa|u-v|^2}{2} - \frac{(\sqrt\varkappa +
\iota/\sqrt\varkappa)^2}{2}
\end{equation}
для всех $ u \in U\setminus B $. Впрочем, если $ u \in B $, то есть $
|u-v|\leqslant 1 $, то 
\[
    \iota < \frac{(\sqrt\varkappa + \iota/\sqrt\varkappa)^2}{2} -
    |u-v|^2\frac{\varkappa}{2},
\]
откуда благодаря неравенству $ f(u) \geqslant f(v) - \iota $ снова выполняется
\eqref{eq:7}. 

Из неравенства \eqref{eq:7} следует, что для любых $ u \in M(v) $  
\[
    \frac{\varkappa|u-v|^2}{2} - \frac{(\sqrt\varkappa +
    \iota/\sqrt\varkappa)^2}{2} \leqslant f(u) - f(v) \leqslant 0,
\]
или $ |u-v| \leqslant 1 + \iota/\varkappa $. Ограниченность $ M(v) $ доказана.
Имеем компактное множество Лебега. Тогда функцией $ f(x) $ достигается её нижняя
грань. Но строго выпуклая функция может иметь лишь одну точку минимума.
\end{proof}

