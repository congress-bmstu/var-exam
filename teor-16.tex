\que{Основная лемма вариационного исчисления (с доказательством), её обобщение.}

%\begin{lemma}
%	Пусть $g(x)$ --- непрерывная функция на $[a, b]$ и $\int_{a}^{b} g(x) h(x) \, dx = 0$ для любой непрерывно-диффуренцируемой функции $h(x): h(a) = h(b) = 0$, тогда $g(x) \equiv 0$.
%\end{lemma}
%
%\begin{proof}
%	От противного.
%	
%	Пусть $g(x)$ не равна нулю тождественно. Не ограничивая общности, будем считать, что эта функция принимает положительные значения (если $g(x) \leqslant 0$ для всех $x \in [a, b]$, то заменим $g(x)$ на $-g(x)$). Тогда в силу непрерывности $g(x)$ существует точка $x_0 \in (a, b)$ такая, что $g(x_0) > 0$, и интервал $[x_0 - \delta, x_0 + \delta] \subseteq (a, b), \, \delta > 0$ такой, что $g(x) \geqslant \frac{g(x_0)}{2}$ для любого $x \in [x_0 - \delta, x_0 + \delta]$. Пусть теперь $h(x)$ --- непрерывно-дифференцируемая функция, причем:
%	\begin{equation*}
	%		\begin{cases}
		%			h(x) \equiv 0, \quad x \not \in (x_0 - \delta, x_0 + \delta); \\
		%			h(x) > 0, \quad x \in (x_0 - \delta, x_0 + \delta).
		%		\end{cases}
	%	\end{equation*}
%	Тогда $\int_{a}^{b} g(x) h(x) \, dx = \int_{x_0 - \delta}^{x_0 + \delta} g(x) h(x) \, dx \geqslant \frac{g(x_0)}{2} \int_{x_0 - \delta}^{x_0 + \delta} h(x) \, dx > 0$, что приводит к противоречию с условием теоремы. 
%	
%	В качестве примера функции $h(x)$, удовлетворяющей записанным выше условиям можно взять 
%	\begin{equation*}
	%		h(x) = \begin{cases}
		%			(x - (x_0 - \delta))^2 (x - (x_0 + \delta))^2, \quad &x \in (x_0 - \delta, x_0 + \delta); \\
		%			0, \quad &x \not \in (x_0 - \delta, x_0 + \delta).
		%		\end{cases}
	%	\end{equation*}
%\end{proof}