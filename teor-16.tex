\que{Основная лемма вариационного исчисления (с доказательством), её обобщение.}

\que{Основная лемма вариационного исчисления (с доказательством), её обобщение.}

\begin{lemma}[основная лемма вариационного исчисления]
	Пусть $\varphi(x)$ -- фиксированнная непрерывная на отрезке $[a, b]$ функция, и для всякой
	непрерывно-дифференцируемой функции $h(x)$ такой, что $h(a) = h(b) = 0$, имеет место 
	\begin{equation}\label{que-16:lemma-eq}
		\int\limits_a^b \varphi(x) h(x) \, dx = 0,
	\end{equation}
	тогда $\varphi(x) \equiv 0$.
\end{lemma}
\begin{proof}
	Пусть $\varphi(x)$ -- не равна нулю тождественно. Не ограничивая общности, будем считать, что
	эта функция принимает положительные значения: $\exists x_0: \varphi(x_0) > 0$.
	Тогда в силу непрерывности,
	\[
	\exists \delta > 0 : \forall x \in (x_0 - \delta, x_0 + \delta) : \varphi(x) \geqslant \dfrac{\varphi(x_0)}{2}.
	\]
	
	Если условие \eqref{que-16:lemma-eq} на интергал выполнено для любой
	непрерывно-дифференцируемой функции $h(x)$,
	то выполнено и для таких <<шапочек>>:
	\[
	h(x) = \begin{cases}
		0, &x \notin (x_0 - \delta, x_0 + \delta) \\
		>0, &x \in (x_0 - \delta, x_0 + \delta)
	\end{cases}
	\]
	Тогда 
	\[
	\int\limits_a^b \varphi(x) h(x) \, dx = \int\limits_{x_0-\delta}^{x_0+\delta} \varphi(x) h(x)\, dx \geqslant \dfrac{\varphi(x_0)}{2} \int\limits_{x_0-\delta}^{x_0 + \delta} h(x) \, dx > 0,
	\]
	что означает противоречие условию теоремы, а значит она верна согласно законам формальной логики.
\end{proof}


Не уверен, что может называться обобщением основной леммы вариационного исчисления, возможно, 
следующая лемма (источник: Гельфанд-Фомин, стр. 17 -- лемма 2): 
\begin{lemma}[<<обобщение>> основной леммы]
	Если 
	\[
	\int\limits_a^b \left[ a(x) h(x) + b(x) h'(x) \right] \, dx = 0
	\]
	для каждой функции $h(x)$ из $D_1$ такой, что $h(a) = h(b) = 0$, то
	$b(x)$ -- дифференцируема и $a(x) - b'(x) = 0$.
\end{lemma}